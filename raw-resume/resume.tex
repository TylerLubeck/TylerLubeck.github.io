%%%%%%%%%%%%%%%%%%%%%%%%%%%%%%%%%%%%%%%%%
% Friggeri Resume/CV
% XeLaTeX Template
% Version 1.0 (5/5/13)
%
% This template has been downloaded from:
% http://www.LaTeXTemplates.com
%
% Original author:
% Adrien Friggeri (adrien@friggeri.net)
% https://github.com/afriggeri/CV
%
% License:
% CC BY-NC-SA 3.0 (http://creativecommons.org/licenses/by-nc-sa/3.0/)
%
% Important notes:
% This template needs to be compiled with XeLaTeX and the bibliography, if used,
% needs to be compiled with biber rather than bibtex.
%
%%%%%%%%%%%%%%%%%%%%%%%%%%%%%%%%%%%%%%%%%

\documentclass[print]{friggeri-cv} % Add 'print' as an option into the square bracket to remove colors from this template for printing

\usepackage[none]{hyphenat}
\begin{document}

\header{Tyler}{Lubeck}{} % Your name and current job title/field

%----------------------------------------------------------------------------------------
%	SIDEBAR SECTION
%----------------------------------------------------------------------------------------

\begin{aside} % In the aside, each new line forces a line break
\section{Contact}
\href{mailto:tyler@tylerlubeck.com}{tyler@tylerlubeck.com}
\href{http://www.tylerlubeck.com}{www.tylerlubeck.com}
\href{http://www.github.com/TylerLubeck}{github.com/TylerLubeck}
\href{https://www.linkedin.com/in/tylerlubeck}{linkedin.com/in/tylerlubeck}
~
(813) 469 1499
\section{Languages}
Python
C, C{}\texttt{++}, C\#
Javascript, Java
\section{Technologies}
Django, Pyramid, Node.js
\section{Tools}
Git, Docker, Consul
Hashicorp Vault
Apache SOLR
\end{aside}

%----------------------------------------------------------------------------------------
%	WORK EXPERIENCE SECTION
%----------------------------------------------------------------------------------------

\section{Experience}

\begin{entrylist}
\entry
{August '15 - Now}
{SurveyMonkey \normalfont{-- \href{http://www.surveymonkey.com}{www.surveymonkey.com}}}
{San Mateo, CA}
{\emph{Software Engineer II}
\begin{itemize}
\item Implement and maintain a SOLR Cloud cluster for the company
\item Implement and maintain the development stack used by the engineering org
\item Lead engineering onboarding for new hires
\item Help lead efforts to implement consistent monitoring and logging infrastructure across the company
\item Implement features across the stack, ranging from MSSQL stored procedures to Python and React based client side applications
\end{itemize}}
%------------------------------------------------
\entry
{Summer '14}
{Microsoft \normalfont{-- \href{http://www.microsoft.com}{www.microsoft.com}}}
{Redmond, WA}
{\emph{Software Development Engineer Intern}
\begin{itemize}
\item Created an API in Sharepoint's Client Side Object Model to manipulate files and folders from both C\# and Javascript 
\item Built a modular file tree control in Javascript, allowing for asynchronous data loading
\item Gained experience working within a large codebase and a complex build system
\end{itemize}}
%------------------------------------------------
\entry
{Summer '13}
{Pegasystems Inc. \normalfont{-- \href{http://www.pega.com}{www.pega.com}}}
{Cambridge, MA}
{\emph{Systems Architect Intern} 
\begin{itemize}
\item Developed software to allow employees to catalog receipts and create expense reports via email
\item Worked with PegaRULES Process Commander to build in-house expense reporting software
\end{itemize}}
%------------------------------------------------
\entry
{2012 -- 2015}
{Tufts University}
{Medford, MA}
{\emph{Co-Instructor: Web Engineering \\Teaching Fellow: Introduction  to Computer Science \\ Teacher's Assistant: Web Programming} 
\begin{itemize}
\item Selected as Head TA for Web Programming course, which entailed supervising a team of 5, planning and implementing review sessions, and holding office hours for 100 students.
\item Taught introductory C{}\texttt{++} lab sections to multiple classes of 20-25 students each.
\item Selected as Teaching Fellow for Introduction to Computer Science, which entailed supervising a team of 35, as well as grading assignments and holding tutoring sessions.
\end{itemize} }
%------------------------------------------------
\entry
{Summer '12}
{Accusoft \normalfont{-- \href{http://www.accusoft.com}{www.accusoft.com}}}
{Tampa, FL}
{\emph{Software Development Intern -- Imaging SDK's}
\begin{itemize}
\item Upgraded sample code projects in VB6, C{}\texttt{++}, and C\# adding new product version functionality and fixing defects
\item Wrote Python scripts for updating MS Visual Studio project files to new versions and for verifying and updating source code copyright markings
\item Authored white paper and accompanying application integrating Barcode Xpress SDK with ImageGear SDK Suite
\end{itemize}}
%------------------------------------------------
\end{entrylist}


%----------------------------------------------------------------------------------------
%	PERSONAL PROJECTS SECTION
%----------------------------------------------------------------------------------------

\section{Personal Projects}

\begin{entrylist}
%------------------------------------------------
\entry
{2013 -- 2015}
{Halligan Helper}
{\href{http://www.halliganhelper.com}{www.halliganhelper.com}}
{\emph{Computer Science Department Resource Tracker}
\begin{itemize}
\item Designed and implemented independent project to collect, track, and graph anonymized computer usage data for the Tufts Computer Science Department
\item Utilized Websockets to build a real-time dashboard to display tutoring wait time
\item Successfully documented and transitioned the project to a new owner after graduating
\end{itemize}}
%------------------------------------------------
\entry
{2014 -- 2015}
{Tufts Marauder's Map}
{}
{\emph{Indoor Positioning System}
\begin{itemize}
\item Engineered system to compute indoor position based on known wireless access points
\item Utilized a kNN machine learning algorithm to determine your position within a building based on known positions of visible wireless access points
\end{itemize}}
\end{entrylist}

\section{Publications}
\begin{entrylist}
\entry
{August '15}
{Device-agnostic Wi-Fi fingerprint positioning for consumer applications}
{IEEE}
{Presented at PIMRC, paper available at \href{http://ieeexplore.ieee.org/document/7343660/}{http://ieeexplore.ieee.org/document/7343660/}}
\end{entrylist}


%----------------------------------------------------------------------------------------
%	EDUCATION SECTION
%----------------------------------------------------------------------------------------

\section{Education}
\begin{entrylist}
%------------------------------------------------
\entry
{2011 -- 2015}
{Bachelor of Science, School of Engineering at Tufts University}
{Medford, MA}
{Major in Computer Science, Minor in Entrepreneurial Leadership}


\end{entrylist}


\end{document}
